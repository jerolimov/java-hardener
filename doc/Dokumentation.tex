%================================================================================
% Erstellt am: 		10.07.2008
% Überarbeitet am:	08.07.2009
% Autor:			Holger Fischer
%
% Kann frei für Bachelor-/Diplom-/Masterarbeiten verwendet werden.
% Viel Erfolg!!!
%=====================================\textbf{}===========================================

%Einbinden der Datei header.tex; diese enthält alle verwendeten Pakete,
%sowie Änderungen am Layout
%!TEX root = ../Dokumentation.tex

%Da Latex für englischsprachige Texte ausgerichtet ist,
%wird als Dokumentenklasse das "`scrbook"' von Markus Kohm verwendet.
%Dieses ist für deutschsprachige Texte ausgelegt.
%BCOR12mm: 12mm Bindekorrektur (Verbreiterung des linken Randes)
%DIV11: entspricht in etwas der geforderten Textgröße und Seitenränder
%titlepage: eine Titelseite wird verwendet
%a4paper: DIN A4
%oneside: für eine spätere einseitige Bedruckung 
%Original: \documentclass[BCOR12mm,DIV11,titlepage,a4paper,oneside]{scrbook}
\documentclass[DIV11,titlepage,a4paper,oneside]{scrbook}

\textheight = 23cm

%Paket für deutsche Silbentrennung etc.
\usepackage{ngerman}

%Paket für Zeichenkodierung, entspricht UTF-8
\usepackage[utf8x]{inputenc}

%Paket das die Ausgabefonts definiert
\usepackage[T1]{fontenc}

%Paket für das Einbinden von Grafiken über die figure-Umgebung
\usepackage{graphicx}

%Paket verhindert floating
\usepackage[section]{placeins}

%Paket zum Ändern der Listenabstände
\usepackage{enumitem}
\setlist[itemize]{itemsep=0mm}
\setlist[enumerate]{itemsep=0mm}

%Paket zum Ändern der Kopf- und Fußzeile
\usepackage{fancyhdr}

%fancypagestyle{fancy}
\renewcommand{\headrulewidth}{0pt}
\fancyhf{}
%\fancyfoot[L]{FANCY}
\fancyfoot[R]{\thepage}

\fancypagestyle{plain}{
	\renewcommand{\headrulewidth}{0pt}
	\fancyhf{}
%	\fancyfoot[L]{PLAIN}
	\fancyfoot[R]{\thepage}
}

\pagestyle{plain}

%Abbildungsnummerierung ändern (abhängig von chapter, z.B. Abbildung 1.1)
\renewcommand*{\thefigure}{\thechapter.\arabic{figure}}
%Tabellennummerierung ändern (abhängig von chapter, z.B. Tabelle 1.1)
\renewcommand*{\thetable}{\thechapter.\arabic{table}}

%Paket, um ein Glossar/Abkürzungsverzeichnis anzulegen
\usepackage{nomencl}
\let\abbrev\nomenclature
%Der Name wird in Glossar geändert
\renewcommand{\nomname}{Glossar (optional)}
%Definiert die Aufteilung im Glossar zwischen Begriffen und Erläuterung
\setlength{\nomlabelwidth}{.25\hsize}
%Definiert die Punktelinien im Glossar
\renewcommand{\nomlabel}[1]{#1 \dotfill}
\setlength{\nomitemsep}{-\parsep}
%Veranlasst die Erstellung des Glossars
\makenomenclature

%Einrückungen nach Absätzen und Grafiken verhindern
\setlength{\parindent}{0pt}

%Verhindern, dass eine neue Seite für ein einzelnes Wort/Zeile verwendet wird
\clubpenalty = 10000 % schliesst Schusterjungen aus 
\widowpenalty = 10000 % schliesst Hurenkinder aus (keine Beleidigung, sondern wirklich ein Fachbegriff)

%Paket für ein deutsches Literaturverzeichnis
\usepackage{bibgerm}

%Paket für die Verwendung von URLs durch den Befehl \url{}
\usepackage{url}

%Paket für Zeilenabstand (onehalfspace, singlespace)
\usepackage{setspace}

%Paket zur Erzeugung von Anführungszeichen durch \enquote{Text}
\usepackage[ngerman]{babel}
\usepackage[babel, german=quotes]{csquotes}

%Paket für farbigen Text
%black,white,green,red,blue,yellow,cyan,magenta
\usepackage{color}

%Paket für farbigen Hintergrund für Verbatim-Umgebung (Quelltext-Umgebung)
\usepackage{fancyvrb}
\usepackage{verbatim,moreverb}

%Grauton für Quelltext-Umgebung definieren 80% Grau
\definecolor{source}{gray}{.90}

%Paket für Quelltext-Umgebung
\usepackage{listings}

%Paket für Positionierung der Objekte ohne Float (Verwendungsbsp.: \begin{figure}[H])
\usepackage{float}

%Paket für Tabellen über die gesamte Breite
\usepackage{tabularx}

%Paket für Definitionen, etc.
\usepackage{amsthm}

%Paket um Grafiken zu erstellen.
\usepackage{tikz}
\usetikzlibrary{arrows,positioning} 

%Paket zur Erzeugung von Hyperrefs und PDF Informationen
\usepackage[pdftex,plainpages=false,pdfpagelabels,
            pdftitle={Projektdokumentation WBA2},
            pdfauthor={Christoph Jerolimov}
            ]{hyperref} 

\begin{document}

%=== Einleitung ======================================================
%Seitennummerierung Abstract bis einschließlich Inhaltsverzeichnis
\frontmatter 

%Einbinden der Titelseite
%!TEX root = ../Dokumentation.tex

\begin{titlepage}

\begin{center}

%Logo der Fachhochschule Köln
\begin{figure}[!ht]
	\centering
		\includegraphics[natwidth=920pt, natheight=95pt, width=1.0\textwidth]{tex/logoheader.pdf}
\end{figure}

\vspace{5.0cm}

\begin{Huge}
	\textbf{WPF Compiler und Interpreter:}\\
	\vspace{0.1cm}
	\textbf{Java-Hardener}\\
\end{Huge}

\vspace{0.8cm}

\begin{LARGE}
	Projektdokumentation über einen Java-Postprozessor\\
	\vspace{0.1cm}
	zur automatisierten Bytecode-Manipulation\\
	\vspace{0.1cm}
	zur Reduzierung von \texttt{NullPointerException}s.\\
\end{LARGE}

\vspace{4.5cm}

\begin{tabular}{rl}
        Dozent:  &  Prof. Dr. Erich Ehses\\
       			 &  \small Fachhochschule Köln \\[1.0em]
\end{tabular}

\vspace{2.0cm}

\begin{large}
	ausgearbeitet von\\
	\vspace{0.2cm}
\end{large}

\begin{Large}
	Christoph Jerolimov, Matrikelnr. 11084742\\
	\vspace{0.5cm}
	Sommersemester 2013
\end{Large}

\end{center}

\end{titlepage}


%Zeilenabstand für das Inhaltsverzeichnis 1 fach
\singlespacing
%Einbinden des Inhaltsverzeichnis
%!TEX root = ../Dokumentation.tex

\cleardoublepage
\pagenumbering{gobble}
\tableofcontents
\cleardoublepage
\pagenumbering{arabic}


%Zeilenabstand für den Hauptteil ist 1,5 fach
\onehalfspacing

%=== Hauptteil =======================================================
%Seitennummerierung des Hauptteils
\mainmatter
	%Die Zähler für Tabellen und Abbildungen werden zurückgesetzt, damit
	%in jedem Kapitel die Nummerierung neu beginnt
	\setcounter{table}{1}
	\setcounter{figure}{1}
	%Einbinden des ersten Kapitels
	%!TEX root = ../Dokumentation.tex

\chapter{Die Idee}

TODO
 

	%Die Zähler für Tabellen und Abbildungen werden zurückgesetzt, damit
	%in jedem Kapitel die Nummerierung neu beginnt
	\setcounter{table}{1}
	\setcounter{figure}{1}
 	%Einbinden des ersten Kapitels
	%!TEX root = ../Dokumentation.tex

\chapter{Analyse}

\section{Problemstellung}

Wie in der Einführung beschrieben, können Objektaufrufe, z.B. durch
Methoden- und Variablenaufrufe (lesend und schreibend), auf \texttt{NULL}
durch vorheriges Prüfen gesichert werden.
Auch andere Fälle, etwa der Zugriff auf Arrays (\texttt{[index]}-Zugriff oder \texttt{.length}) kann
zu NPE-Ausnahmefehlern führen. Nicht alle diese Anwendungsfälle werden
in diesem Prototyp umgesetzt sollen aber wenigstens in dieser Einführung angesprochen werden.

Problematisch sind insbesondere verkettete Aufrufe (vgl. Listing 2.1).
So müssen die Zwischenergebnisse etwa in lokalen Variablen gespeichert werden (vgl. Listing 2.2)
oder die Aufrufe wiederholt werden, wenn diese in umgebende Bedingungen eingebaut werden (vgl. Listing 2.3).
Letzteres würde jedoch nicht nur die Performance negativ beeinflussen, sondern
könnte bei inmutablen Zugriffen auch zu Fehlerhaften Programmabläufen führen.

\begin{lstlisting}[basicstyle=\ttfamily,numbers=left,numberstyle=\footnotesize\ttfamily,backgroundcolor=\color{source}]
Deque<Map<String, Integer>> example = null;
int size = example.getFirst().get("size");
\end{lstlisting}
\centerline{Listing 2.1: Beispiel für verkette Aufrufe}

\vspace{0.3cm}

\begin{lstlisting}[basicstyle=\ttfamily,numbers=left,numberstyle=\footnotesize\ttfamily,backgroundcolor=\color{source}]
Deque<Map<String, Integer>> example = null;
Map v1 = example.getFirst();
Integer v2 = v1.getSize("size");
int size = v2 != null ? v2.intValue() : 0;
\end{lstlisting}
\centerline{Listing 2.2: Umwandlung verketteter Aufrufe in lokale Variablen}

\vspace{0.3cm}

\begin{lstlisting}[basicstyle=\ttfamily,numbers=left,numberstyle=\footnotesize\ttfamily,backgroundcolor=\color{source}]
Deque<Map<String, Integer>> example = null;
int size = 0;
if (example != null &&
	example.getFirst() != null &&
	example.getFirst().get("size") != null) {
	size = example.getFirst().get("size");
}
\end{lstlisting}
\centerline{Listing 2.3: Verkettete Aufrufe umfasst mit NULL-Prüfungen}

\vspace{0.3cm}

Autoboxing bezeichnet die mit Java 1.5 eingeführte automatische Umwandlung zwischen
primitiver Datentypen sowie deren Wrapper-Typen. Diese implizite Umwandlung wird
durch zusätzliche Methodenaufrufe durch den Compiler eingewebt und ist für den
Java-Interpreter nicht von normalen Aufrufen zu unterscheiden.

Für die Manipulation des Bytecodes zur Verbesserung der Fehlertoleranz sollte
dies ebenfalls keinen Unterschied bieten.


\section{Bytecode-Analyse}

Mithilfe des im ASM enthaltenen \texttt{Textifier} Anwendung können verschiedene Lösungswege
deassembliert und analysiert werden. Zum einfacheren Aufruf wurde ein kleines Shell-Script
(siehe textifier) erstellt. Mitdessen Hilfe wurden etwa für das in Listing 2.4 angegebene
Java-File die in 2.5 angegebene Ausgabe erzeugt.

Der Aufruf erfolgt über den Scriptnamen gefolgt von einer Java-Bytecode-Datei:

\vspace{0.3cm}

\texttt{./textifier target/test-classes/de/fhkoeln/.../testcases/Test1.class}


\subsection{Ausgangsbasis}

\begin{lstlisting}[basicstyle=\ttfamily,numbers=left,numberstyle=\footnotesize\ttfamily,backgroundcolor=\color{source}]
public class Test1 {
	public int getStringLength(Map map, String key) {
		return ((String) map.get(key)).length();
	}
}
\end{lstlisting}
\centerline{Listing 2.4: Beispiel Sourcecode ohne NULL-Prüfung}

\vspace{0.3cm}


\begin{lstlisting}[basicstyle=\ttfamily,numbers=left,numberstyle=\footnotesize\ttfamily,backgroundcolor=\color{source}]
public class de/fhkoeln/.../testcases/Test1 {
  public getStringLength(LMap;LString;)I
    ALOAD 1
    ALOAD 2
    INVOKEINTERFACE Map.get (LObject;)LObject;
    CHECKCAST String
    INVOKEVIRTUAL String.length ()I
    IRETURN
    MAXSTACK = 2
    MAXLOCALS = 3
}
\end{lstlisting}
\centerline{Listing 2.5: Bytecode Auszug für Listing 2.4}

\vspace{0.3cm}

Im Folgenden sollen die Unterschiede aufgezeigt werden, wenn man diese ursprüngliche
Version mit einer \texttt{NPE}-gesicherte Versionen vergleicht. Die dafür angelegten Klassen
befinden sich im Quellcode-Ordner \texttt{test} innerhalb des Java-Packages \texttt{de.fhkoeln...analysebytecode}.

\vspace{0.3cm}


\subsection{Bedingte Ausführung}

Durch eine \texttt{NULL}-Prüfung, etwa mit einem Bedingungsoperator \texttt{?:},
fügt der Compiler zwei Labels (Ziele für Springmarken) ein und prüft
anschließend die aktuell auf dem Stack liegende Variable auf \texttt{NULL}
(vgl. Listing 2.6 Zeile 2). Ergibt etwa die Prüfung von
\texttt{e != null ? e.toString() : null}, dass die Variable \texttt{NULL} ist,
so springt die Ausführung zur angegebenen Sprungmarke (hier \texttt{L0})
und fügt eine \texttt{NULL}-Referenz auf den Stack hinzu (Zeile 7).
Falls die \texttt{NULL}-Prüfung jedoch negativ ausviel, wird die Ausführung fortgesetzt
und der eigentliche Methodenaufruf durchgeführt (\texttt{INVOKEVIRTUAL} in Zeile 4).
Um anschließend den nicht benötigten alternativen Zweig der Anwendung zu gehen,
wird dieser mithilfe eines \texttt{GOTO}s (hier zur Sprungmarke \texttt{L1}) übersprungen.

\begin{lstlisting}[basicstyle=\ttfamily,numbers=left,numberstyle=\footnotesize\ttfamily,backgroundcolor=\color{source}]
    ALOAD 2 /* entry */
    IFNULL L0
    ALOAD 2 /* entry */
    INVOKEVIRTUAL String.toString ()LString;
    GOTO L1
   L0
    ACONST_NULL
   L1
\end{lstlisting}
\centerline{Listing 2.6: Auszug ASM für Null-Prüfung mit Bedingungsoperator}

\vspace{0.3cm}


\subsection{Try-Catch}

Eine weitere Möglichkeit wäre die mögliche Ausnahmebehandlung von dem eingebauten
try-catch Mechanismus behandeln zu lassen und einen entsprechenden Block um
den möglicherweise zu fehlern führenden Aufruf zu erstellen.

Für dieses Vorgehen wird eine zusätzliche lokale Variable benötigt,
welche im Fehlerfall mit einem Defaultwert gefüllt wird (vgl. Listing 2.7).

\begin{lstlisting}[basicstyle=\ttfamily,numbers=left,numberstyle=\footnotesize\ttfamily,backgroundcolor=\color{source}]
int l;
try {
	l = entry.length();
} catch (NullPointerException e) {
	l = 0;
}
\end{lstlisting}
\centerline{Listing 2.7: Beispiel Null-Prüfung mit try-catch}

\vspace{0.3cm}

Der daraufhin entstehende Bytecode speichert das Ergebnis des Originalaufrufs
in einer lokalen Variable (Listing 2.8 Zeile 4 und 5). Sollte es während
dieses Aufrufs zu einer Fehlerbehandlung kommen, wird diese Variable mit
einer \texttt{NULL}-Referenz überschrieben (Zeile 10 und 11).


\begin{lstlisting}[basicstyle=\ttfamily,numbers=left,numberstyle=\footnotesize\ttfamily,backgroundcolor=\color{source}]
    TRYCATCHBLOCK L0 L1 L2 NullPointerException
   L0
    ALOAD 2
    INVOKEVIRTUAL String.length ()I
    ISTORE 3
   L1
    GOTO L3
   L2
    ASTORE 4
    ICONST_0
    ISTORE 3
   L3
\end{lstlisting}
\centerline{Listing 2.8: Auszug ASM für Null-Prüfung mit try-catch}

\vspace{0.3cm}

Insgesamt fällt auf, dass dieser Code bereits bei diesem einfachen Beispiel deutlich
mehr Intruktionen beinhaltet als die zuvor genannte Bedingungsoperator-Variante.
Gleichzeitig wird für quasi jeden Methodenaufruf eine zusätzliche lokale Variable
benötigt. (Ggf. könnten diese auf eine Varaible je Datentyp kombiniert werden.)

Darüber hinaus würde diese Variante nicht nur unmittelbare NullPointerExceptions
abfangen sondern auch Fehler welche innerhalb der Methode ausgeführt werden und
ggf. gar nicht vom java-hardener manipuliert wurden.


	
	%Die Zähler für Tabellen und Abbildungen werden zurückgesetzt, damit
	%in jedem Kapitel die Nummerierung neu beginnt
	\setcounter{table}{1}
	\setcounter{figure}{1}
	%Einbinden des zweiten Kapitels
	%!TEX root = ../Dokumentation.tex

\chapter{Kapitel 3}

TODO


	%Die Zähler für Tabellen und Abbildungen werden zurückgesetzt, damit
	%in jedem Kapitel die Nummerierung neu beginnt
	\setcounter{table}{1}
	\setcounter{figure}{1}
	%Einbinden des zweiten Kapitels
	%!TEX root = ../Dokumentation.tex

\chapter{Kapitel 4}

TODO


	%Die Zähler für Tabellen und Abbildungen werden zurückgesetzt, damit
	%in jedem Kapitel die Nummerierung neu beginnt
	\setcounter{table}{1}
	\setcounter{figure}{1}
	%Einbinden des zweiten Kapitels
	%!TEX root = ../Dokumentation.tex

\chapter{Kapitel 5}

TODO


	%Die Zähler für Tabellen und Abbildungen werden zurückgesetzt, damit
	%in jedem Kapitel die Nummerierung neu beginnt
	\setcounter{table}{1}
	\setcounter{figure}{1}
	%Einbinden des zweiten Kapitels
	%!TEX root = ../Dokumentation.tex

\chapter{Fazit}

Zum Ende eines jeden Projektes sollte die Entwicklung
und der Abschluss noch einmal kritisch hinterfragt und den
vorher gesetzten Zielen gegenübergestellt werden.

\section{Projektstatus}

\subsubsection{Positive Aspekte}

\subsubsection{Negative Aspekte}

\section{Reflektion und Ausblick}


%Zeilenabstand 1 fach für die Verzeichnisse
\singlespacing
%Einbindne der Verzeichnisse
\include{tex/verzeichnisse}

%Zeilenabstand 1,5 fach für den Eid
\onehalfspacing
%Einbinden des Eides
%!TEX root = ../Dokumentation.tex

\chapter*{Eidesstattliche Erklärung}
\addcontentsline{toc}{chapter}{Eidesstattliche Erklärung}
Ich versichere, die von mir vorgelegte Arbeit selbständig verfasst zu haben.\\ \\
Alle Stellen, die wörtlich oder sinngemäß aus veröffentlichten oder nicht veröffentlichten Arbeiten anderer entnommen sind, habe ich als entnommen kenntlich gemacht. Sämtliche Quellen und Hilfsmittel, die ich für die Arbeit benutzt habe, sind angegeben.\\ \\
Die Arbeit hat mit gleichem Inhalt bzw. in wesentlichen Teilen noch keiner anderen Prüfungsgbehörde vorgelegen.
\vspace{1.5cm}
\\
Gummersbach, 31. Juli 2013
\vspace{3cm}
\\
Christoph Jerolimov


\end{document}