%!TEX root = ../Dokumentation.tex

\chapter{Abstrakt}

\texttt{NullPointerException}s (im Folgendem \texttt{NPE}) sind ein (üblicherweise leicht lösbares)
Standardproblem der Softwareentwicklung und treten in der Programmiersprache Java auf,
wenn Methoden- oder Attribut-Zugriffe
auf \texttt{null}-Object erfolgen\footnote{Dadrüber hinaus kann eine NPE auch
noch in anderen Fällen geworfen werden. Vgl. http://www.java-blog-buch.de/0503-nullpointerexception/, abgerufen am 31.07.2013}.

Die Behandlung solcher ungültiger Aufrufe ist grundsätzlich abhängig von der
Programmiersprache (Definition) bzw. der Laufzeitumgebung. So können entsprechende Zugriffe
zum Absturz des Programms führen, wie in Java zum werfen einer entsprechender
Ausnahme oder, wie etwa in Objective-C, ignoriert werden\footnote{Weitereführende Informationen zur Sprache Objective-C: http://developer.apple.com/library/mac/documentation/Cocoa/Conceptual/ProgrammingWithObjectiveC/, abgerufen am 31.07.2013}.

Diese fehlertolerantere Version von Objective-C soll hier für die Programmiersprache Java
nachgebildet werden
und durch eine automatisierte manipulation des Java-Bytecodes erreicht werden.
Wie in der Vorlage müssen, soweit dies deklariert wurde, Methoden einen Rückgabewert liefern.
Hier werden, analog zu Objective-C, möglichst neutrale Werte gewählt:
\texttt{false} für boolsche Ausdrücke, \emph{Null (0)} für Zahlen sowie \texttt{NULL}-Referenzen für Objekte.

Die beiden folgenden zwei Anwendungsfälle (vgl. Listing 1.1 und 1.2) verdeutlichen
die vereinfachte Lesbarkeit und würden ohne Bytecode-Manipulation jeweils zu einer
zu \texttt{NPE} führen.

\begin{lstlisting}[basicstyle=\ttfamily,numbers=left,numberstyle=\footnotesize\ttfamily,backgroundcolor=\color{source}]
List nullList = null;
System.out.println("List size: " + nullList.size());
\end{lstlisting}
\centerline{Listing 1.1: Beispiel für einen Null-Zugriff mit erwartetem Integer-Ergebnis}

\vspace{0.3cm}

\begin{lstlisting}[basicstyle=\ttfamily,numbers=left,numberstyle=\footnotesize\ttfamily,backgroundcolor=\color{source}]
List nullList = null;
if (!nullList.isEmpty()) {
	// Will run this code also if the nullList is null...
}
\end{lstlisting}
\centerline{Listing 1.2: Beispiel für einen Null-Zugriff mit erwartetem Boolean-Ergebnis}

\vspace{0.3cm}

Im Folgenden werden verschiedene technische Möglichkeiten untersucht,
sowie deren prototypische Umsetzung beschrieben.

Zur besseren Lesbarkeit wird auf Java-Packagenamen (etwa \texttt{java.lang}) sowie Generics
in allen Java und Bytecode (Assembler ähliche Schreibweise) Darstellungen verzichtet.
Alle angesprochenen Dateien sind Bestandteil des zugehörigem Quellcodeprojekts\footnote{U.a. zu finden auf GitHub: https://github.com/jerolimov/java-hardener}.

