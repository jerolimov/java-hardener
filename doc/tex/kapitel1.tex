%!TEX root = ../Dokumentation.tex

\chapter{Die Idee}

\texttt{NullPointerException}s (NPE) sind ein klassisches Problem der Softwareentwicklung
und treten in der Programmiersprache Java auf wenn Methoden- oder Attribut-Zugriffe
auf \texttt{null}-Object erfolgen\footnote{Dadrüber hinaus kann eine NPE auch
noch in anderen Fällen geworfen werden. Vgl. http://www.java-blog-buch.de/0503-nullpointerexception/}.

Die Behandlung solcher ungültiger Aufrufe ist grundsätzlich abhängig von der
Programmiersprache und der Laufzeitumgebung. So können entsprechende Zugriffe
zum Absturz des Programms führen, wie in Java zum werfen einer entsprechender
Ausnahme oder, wie etwa in Objective-C\footnote{Vgl. http://developer.apple.com/library/mac/documentation/Cocoa/Conceptual/ProgrammingWithObjectiveC/}, ignoriert werden.

Diese fehlertolerantere Version von Objective-C soll hier nachgebildet werden
und durch eine automatisierte manipulation des Java-Bytecodes erreicht werden.
Wie in der Vorlage müssen entsprechende Methoden immer einen Rückgabewert liefern,
hier werden, analog zu Objective-C, möglichst neutrale Werte gewählt:
False für boolsche Ausdrücke, Null für Zahlen und NULL-Referenzen für Objekte

Die beiden folgenden zwei Anwendungsfälle (vgl. Listing 1.1 und 1.2) verdeutlichen
die Einfachheit für den Programmier und würden ohne Bytecode-Manipulation
zu NullPointerExceptions führen.

\begin{lstlisting}[basicstyle=\ttfamily,numbers=left,numberstyle=\footnotesize\ttfamily,backgroundcolor=\color{source}]
List nullList = null;
System.out.println("List size: " + nullList.size());
\end{lstlisting}
\centerline{Listing 1.1: Beispiel für einen Null-Zugriff mit erwartetem Integer-Ergebnis}

\vspace{0.3cm}

\begin{lstlisting}[basicstyle=\ttfamily,numbers=left,numberstyle=\footnotesize\ttfamily,backgroundcolor=\color{source}]
List nullList = null;
if (!nullList.isEmpty()) {
	// Will run this code also if the nullList is null...
}
\end{lstlisting}
\centerline{Listing 1.2: Beispiel für einen Null-Zugriff mit erwartetem Boolean-Ergebnis}

\vspace{0.3cm}

Für die Umsetzung bietet sich die ASM\footnote{Vgl. http://asm.ow2.org/} Bibliothek
an welche für das manipulieren von Java-Bytecodes verschiedene technische Möglichkeiten an,
diese werden im folgendem Untersucht und deren prototypische Umsetzung beschrieben wird.
