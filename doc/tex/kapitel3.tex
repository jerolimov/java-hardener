%!TEX root = ../Dokumentation.tex

\chapter{Umsetzung}

\section{Maven}

Um die Abhängigkeiten mit Maven runterzuladen kann ein entsprechenden IDE-maven-plugin
verwednet werden oder die IDE Konfiguration mit den folgenden Befehlen erzeugt werden:

\begin{lstlisting}[basicstyle=\ttfamily,backgroundcolor=\color{source}]
mvn eclipse:clean eclipse:eclipse -DdownloadSources
mvn idea:clean idea:eclipse
\end{lstlisting}

\vspace{0.3cm}

Zum runterladen der Resourcen und compilieren des Projektes kann
anschließend die IDE verwendet werden oder einer der folgenden Befehle
zum bauen bzw. paketieren der Klassen als JAR-Datei:

\begin{lstlisting}[basicstyle=\ttfamily,backgroundcolor=\color{source}]
mvn compile
mvn test       # Beinhaltet compile
mvn package    # Beinhaltet test
\end{lstlisting}

\vspace{0.3cm}

\section{ASM}

Zur Manipulation von Java Bytecode bietet sich die leichtgewichtige und speziel
dafür entwickelte OpenSource-Bibliothek ASM an. Während der Entwicklung wurden
drei ASM-Libraries mithilfe von Maven eingebunden:

\begin{itemize}
\item Die Kernbibliothek ASM (asm-4.x.jar) bietet Schnittstellen zum Einlesen und Schreiben
		von Class-Dateien mithilfe des Visitor-Patterns.
\item Optional kann ASM durch eine Library zum DOM-basierten Zugriff auf den
		Bytecode erweitert werden (asm-tree-4.x.jar).
\item Häufig verwendete Methoden, etwa zum Ausgeben von Assembler-Code finden
		sich in der ebenfalls optionalen Utility-Erweiterung (asm-util-4.x.jar).
\end{itemize}


\subsection{Visitor Pattern}

Zur Manipulation des Bytecodes verwendet ASM das Visitor Pattern und verschachtelt
dabei drei verschiedene Visitor Schnittstellen (jeweils als Abstrakte Klassen):

\begin{itemize}
	\item \texttt{ClassVisitor} für den Header einer Klasse, Annotations, etc.
			Diese Klasse deligiert den Visitor für Methoden und Klassenvariablen (Fields)
			an neue Instanzen der folgenden Klassen.
	\item \texttt{MethodVisitor} bietet visitor Methoden für die Methoden deklaration
			sowie die enthaltene Implementation (Operationsaufrufe für den virtuellen Java-Prozessor).
	\item \texttt{FieldVisitor} bietet ausschließlich die Möglichkeit auf die deklarierte,
			und ggf. annotierte Klassenvariable zu reagieren.
\end{itemize}

Zum schreiben von Klassen bietet ASM mit der Klasse \texttt{ClassWriter} eine
Implementierung des \texttt{ClassVisitor} welche sein Ergebnis in einen entsprechden
Ausgabekanal schreibt.

Zur Visualisierung des Assembler-Codes bietet sich die Klasse \texttt{TraceClassVisitor}
an welche eine menchenlesbare Ausgabe produziert.

\subsection{Tree / DOM API}

Alternativ zum Visitor Pattern bietet die ASM-Tree Bibliothek einen dadrauf aufbauenden
wahlfreien (DOM-basierten) Zugriff auf den Klassencode.

Dies hat den Vorteil das deutlich komplexere analysen möglich sind und der
Kontext eines Befehles mit betrachtet werden kann.
Jedoch sind solche analysen deutlich Komplexer als diese etwa auf
einem Quellcode-DOM wären da viele Informationen beim reduzieren auf
Assembler-Bytecode verloren gehen.

\section{Umsetzung automatisierte IFNULL-Prüfung}

Nach einer Testumsetzung und verschiedenen Analysemöglichkeiten findet sich
das Ergebnis in den beiden Klassen \texttt{CheckNullClassVisitor} sowie
\texttt{CheckNullMethodVisitor}. Während ersetzter die nötige Schnittstelle für die
\texttt{ClassReader.accept(ClassVisitor classVisitor, int flags)} Methode implementiert
hat diese jedoch keine manipulierende Auswirkung auf den Bytecode. Ihre
einzige Funktion ist es für jede zu prüfende Methode (\texttt{visitMethod})
eine neue Instanz der Klasse \texttt{CheckNullMethodVisitor} zurück zu geben.

Der Methoden-Vistor kümmert sich anschließend um die Prüfung aller \texttt{INVOKE\_\*}
Assembler aufrufe. Hierfür muss die Methode folgende Methode überladen werden:

\texttt{visitMethodInsn(int opcode, String owner, String name, String desc)}

Für nicht behandelete Anwendungsfälle reicht es die Implementierung der Elternklasse
aufzurufen. Wenn stattdessen andere \texttt{visit*} Methoden der Elternklasse aufgerufen
werden, werden diese Methoden an den im Konstruktur übergebenen Visitor übergeben.

Auf diese Art können verschiedene \texttt{MethodVisitor} ineinendder geschachtelt
(chaining) werden und die jeweiligen Teilaufgaben übernehmen. Eine übergebene \texttt{ClassWriter}
Instanz kann etwa die veränderten visit-Aufrufe in Bytecode umwandeln. Vgl. hierzu
auch die Debug-Möglichkeiten im Kapitel Umsetzung ClassLoader.

\subsection{Iteration 1: Grundsätzliches Vorgehen}

Die erste prototische Umsetzung\footnote{Vgl. Projektsourcen - Rev 951f48 \href{https://github.com/jerolimov/java-hardener/blob/951f48194f53baebd0915c01e0ed3cc2596bd0db/src/main/java/de/fhkoeln/gm/cui/javahardener/CheckNullMethodVisitor.java\#L43-66}{CheckNullMethodVisitor.java Zeile 43-66}} der Klasse \texttt{CheckNullMethodVisitor} behandelte Ausschließlich
den Methodenaufruf \texttt{String.length()}. Alle anderen Aufrufe wurden
in dieser Version nicht beachtet. Für die Null-Prüfung wurde der Original Aufruf
in eine Bedingung mit Sprungbefehlen gekapselt.

Um den Original Aufruf nicht zu verändern muss die aktuelle Stackreferenz
auf das Objekt welches die Methode ausführt mittels \texttt{DUP} (vgl. Listing 3.1 Zeile 5)
verdoppelt werden. Diese neue Referenz wird bei der \texttt{NULL}-Prüfung mit
\texttt{IFNULL} wieder vom Stack gelöscht. Ergibt die Prüfung das es sich um
eine \texttt{NULL}-Referenz handelt springt deie Luafzeitumgebung zur
angegebenen Springmarke (hier \texttt{Label fallback}, vgl. Zeile 2, 6 und 10).
Wenn die NULL-Prüfung ergibt das es zu keiner NullPointerException kommen
wird wird in der nächsten Intruktion der Original Aufruf durchgeführt
und hier die super Methode aufgerufen welche die Argumente an den jeweils
nachgeschalteten \texttt{MethodVisitor} übergibt.
Um den im folgendenen beschriebenen alternativen Anwendungspfad nicht
zu durchlaufen wird dieser mit einem \texttt{GOTO} und der Zielsprungmarke
übersprungen.

Falls der Aufruf nicht ausgeführt werden soll, da die aktuelle Pointerreferenz \texttt{NULL}
ist muss dieser Aufruf mithilfe von \texttt{POP} vom Stack entfernt werden.
(Während die duplizierte Adresse von IFNULL aufgebraucht wurde, würde der
eigentliche Aufruf einer Methode die Objektreferenz löschen und durch das Ergebnis
ersetzen.)

Damit die nächsten Instruktionen mit dem erwartetem Ergebnis auf dem Stack
rechnen können muss anschließend nur noch ein Standard-Ergebnis
auf den Stack geschrieben werden. Für die aktuelle Methode (\texttt{String.length()})
bietet sich hierfür die Instruktion \texttt{ICONST\_0} an. Dies fügt ein int 0 dem Stack hinzu.

\begin{lstlisting}[basicstyle=\ttfamily,numbers=left,numberstyle=\footnotesize\ttfamily,backgroundcolor=\color{source}]
public void visitMethodInsn(...) {
	Label fallback = new Label(); Label behind = new Label();

	super.visitInsn(Opcodes.DUP);
	super.visitJumpInsn(Opcodes.IFNULL, fallback);
	super.visitMethodInsn(opcode, owner, name, desc);
	super.visitJumpInsn(Opcodes.GOTO, behind);

	super.visitLabel(fallback);
	super.visitInsn(Opcodes.POP);
	super.visitInsn(Opcodes.ICONST_0);
	super.visitLabel(behind);
}
\end{lstlisting}
\centerline{Listing 3.1: Erste Umsetzung einer automatischen Null-Prüfung mit ASM}

\vspace{0.3cm}


\subsection{Iteration 2: Verfielfältigung}

Bei der zweiten Iteration\footnote{Vgl. Projektsourcen Rev 749111 - \href{https://github.com/jerolimov/java-hardener/blob/749111f5dcc3f71a1d1db5a669591288245e912b/src/main/java/de/fhkoeln/gm/cui/javahardener/CheckNullMethodVisitor.java\#L42-131}{CheckNullMethodVisitor.java Zeile 42-131}}
wurde versucht dieses Vorgehen auch auf andere Methoden anzuwenden und die
jeweiligen Unterschiede zu beleuchten.

Problematisch dabei ist die Reihenfolge des Stacks für den jeweiligen Methodenaufruf.
So liegen auf oberster Position die Argumente und erst \"unter diesem\" die eigentliche
Referenz auf das Objekt wessen Methode aufgerufen werden soll. Um die Objektreferenz
einer \texttt{NULL}-Prüfung mit \texttt{IFNULL} unterzuziehen zu können muss
jedoch diese jedoch oben auf dem Stack aufliegen.

Für Methoden mit nur einem Argument konnte dies noch einfach über das Hintereinander
schalten der beiden Intruktionen \texttt{DUP2} sowie \texttt{POP} sein.
Während der erste Befehl (bei nur einem Argument) die Referenz des Objektes und des Arguments
kopiert, wird die des Arguments anschließend wider entfernt.

Für eine Beispielhafte Implementierung für die Methode \texttt{Map.get(Object)}
muss schließlich nicht nur eine Referenz sondern ebenfalls zwei Referenzen vom
Stack gegen eine NULL-Referenz (anstatt eines int 0) ersetzt werden (Vgl. Listing 3.2).

\begin{lstlisting}[basicstyle=\ttfamily,numbers=left,numberstyle=\footnotesize\ttfamily,backgroundcolor=\color{source}]
super.visitInsn(Opcodes.DUP2);
super.visitInsn(Opcodes.POP);
[...]
super.visitInsn(Opcodes.POP2);
super.visitInsn(Opcodes.ACONST_NULL);
\end{lstlisting}
\centerline{Listing 3.2: Auszug für eine automatischen Null-Prüfung mit einem Argument}

\vspace{0.3cm}

Dieser Mechanismus funktioniert jedoch nur Argumente mit maximal einem Parameter.
Gleichzeitig darf dieser Parameter weder ein \texttt{long} noch ein \texttt{double} sein,
da die \texttt{DUP2} Instruktion den Stack bit-orientiert kopiert\footnote{Vgl. http://en.wikipedia.org/wiki/Java\_bytecode\_instruction\_listings}.

\subsection{Iteration 3: Generalisierung}

Mit der dritten Iteration\footnote{Vgl. Projektsourcen Rev c6e6bd - \href{https://github.com/jerolimov/java-hardener/blob/c6e6bdc7d081eae5e47d2c926073aa3715d908f6/src/main/java/de/fhkoeln/gm/cui/javahardener/CheckNullMethodVisitor.java\#L42-169}{CheckNullMethodVisitor.java Zeile 42-169}}
wurden schließlich die Sonderbehandlungen für \texttt{String.length()} und
\texttt{Map.get(Object)} gegen eine allgemeingültigere Implementierung ersetzt.
Entsprechend der vorhergenannten Einschränkungen können aktuell nur Methoden ohne
bzw. maximal einem Argument so manipuliert werden das es zu keiner NPE kommen kann
(vgl. Listing 3.3).

Über den Rückgabetyp der Funktion kann zusätzlich erkannt werden von
welchem Typ ein entsprechender Eintrag auf dem Stack sein muss.
Auch wenn dieser Unterschied zur Laufzeit vermutlich keine Relevanz hat
wird bei laden der Klasse der Bytecode auf innere Korrektheit geprüft
was den korrekten Datentyp erforderte (vgl. Listing 3.4).

\begin{lstlisting}[basicstyle=\ttfamily,numbers=left,numberstyle=\footnotesize\ttfamily,backgroundcolor=\color{source}]
Type type = Type.getType(desc);
int argumentCount = type.getArgumentTypes().length;
int argumentSize = type.getArgumentTypes()[0].getSize();

if (argumentCount == 0) {
	invokeMethodWithoutArguments(opcode, owner, name, desc);
} else if (argumentCount == 1 && argumentSize < 2) {
	invokeMethodWithOneArgument(opcode, owner, name, desc);
} else {
	super.visitMethodInsn(opcode, owner, name, desc);
}
\end{lstlisting}
\centerline{Listing 3.3: Einschränkung der manipulierbaren Methodenaufrufe}

\vspace{0.3cm}


\begin{lstlisting}[basicstyle=\ttfamily,numbers=left,numberstyle=\footnotesize\ttfamily,backgroundcolor=\color{source}]
private void pushDefault(Type type) {
	switch (type.getSort()) {
	case Type.VOID: break;
	case Type.BOOLEAN: super.visitIntInsn(Opcodes.BIPUSH, 0); break;
	case Type.INT: super.visitInsn(Opcodes.ICONST_0); break;
	[...]
	default: super.visitInsn(Opcodes.ACONST_NULL); break;
	}
}
\end{lstlisting}
\centerline{Listing 3.4: Typabhängiges setzen von Defaultwerten auf den Stack}

\vspace{0.3cm}


\subsection{StackSize und Labels}

Die maximale Stacksize kann surch durch dieses Vorgehen ändern und sollte
im ungünstigsten Fall \emph{1 x Anzahl Methoden ohne Argumente +}
\emph{2 x Anzahl Methoden mit einem Argument} größer werden.
Dieser unwahrscheinliche Fall tritt jedoch nur auf wenn alle Methoden
Ergebnisse (ohne Lokale Zwischenspeicherung in Variablen) aufeinander aufbauen.

Zur Optimierung innerhalb der Java-Laufzeitumgebung beinhalten Methoden
die Informationen wieviele elemente sich maximal auf dem Stack befinden können.
Diese Variable (\texttt{MAXSTACK =} \emph{n}) muss entsprechend der oben
gekannten Berechnungen angepasst werden.

Durch das setzen des \texttt{ClassWriter.COMPUTE\_MAXS} flags beim erzeugen
einer \texttt{ClassWriter} Instanz kann die ASM Bibliothek diese Berechnung
übernehmen. Da der Java-Bytecode beim laden entsprechende Konsistenz-Prüfungen 
vornimmt ist es drigend erforderlich das dieses erneute Berechnung stattfinden.


\section{Umsetzung ClassLoader}

Die ClassLoader-Implementierung \texttt{JHClassLoader} lädt die nötigen Resourcen
von ihrem Parent-ClassLoader und manipuliert den erhaltenen Bytecode-Datenstrom
mithilfe der oben beschriebenen \texttt{CheckNullClassVisitor} Implementierung.

Durch passendes Zusammenstecken der ClassVisitor-chain kann der gegebene
Bytecode-Datenstrom bzw. das manipulierte Bytecode-Ergebnis mithilfe der
Klasse \texttt{TraceClassVisitor} auch ausgegeben werden.


\section{TODO}


\begin{itemize}
\item Shell-Script das Class-Dateien bearbeitet.
\item Ein kleines Shell-Script welches den Classloader setzt (für bestimmte Klassen?
\item zB \texttt{javahardener -Dharden=methodcalls -cp … Main}?
\item oder \texttt{jarh -Dharden=methodcalls beispiel.jar}?
\end{itemize}

