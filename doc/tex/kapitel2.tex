%!TEX root = ../Dokumentation.tex

\chapter{Analyse}

%Um die Komplexität abschätzen zu können bedarf es der Vorherigen Analyse
%möglicher Anwendungsfälle oder Probleme.

\section{Problemstellung}

Wie in der Einführung beschrieben, können Objectaufrufe, z.B. durch
Methoden- und Variablenaufrufe (lesend und schreibend), auf \texttt{NULL}
durch vorheriges Prüfen gesichert werden.
Auch andere Fälle, etwa der Zugriff auf Arrays (\texttt{[index]}-Zugriff oder \texttt{.length}) kann
zu NPE-Ausnahmefehlern führen. Nicht alle diese Anwendungsfälle werden
in diesem Prototypem umgesetzt sollen aber wenigstens in dieser Einführung angesprochen werden.

Problematisch sind insbesondere verkettete Aufrufe (vgl. Listing 2.1).
So müssen die zwischen Ergebnisse etwa in lokalen Variablen gespeichert werden (vgl. Listing 2.2)
oder die Aufrufe wiederholt werden wenn diese in umgebende Bedingungen einzubauen (vgl. Listing 2.3).
Letzteres würde jedoch nicht nur die Performance negativ beeinflussen, sondern
könnte bei inmutablen Zugriffen auch zu Fehlerhaften Programmabläufen führen.

\begin{lstlisting}[basicstyle=\ttfamily,numbers=left,numberstyle=\footnotesize\ttfamily,backgroundcolor=\color{source}]
Deque<Map<String, Integer>> example = null;
int size = example.getFirst().get("size");
\end{lstlisting}
\centerline{Listing 2.1: Beispiel für verkette Aufrufe}

\vspace{0.3cm}

\begin{lstlisting}[basicstyle=\ttfamily,numbers=left,numberstyle=\footnotesize\ttfamily,backgroundcolor=\color{source}]
Deque<Map<String, Integer>> example = null;
Map v1 = example.getFirst();
Integer v2 = v1.getSize("size");
int size = v2 != null ? v2.intValue() : 0;
\end{lstlisting}
\centerline{Listing 2.2: Umwandlung verketteter Aufrufe in lokale Variablen}

\vspace{0.3cm}

\begin{lstlisting}[basicstyle=\ttfamily,numbers=left,numberstyle=\footnotesize\ttfamily,backgroundcolor=\color{source}]
Deque<Map<String, Integer>> example = null;
int size = 0;
if (example != null &&
	example.getFirst() != null &&
	example.getFirst().get("size") != null) {
	size = example.getFirst().get("size");
}
\end{lstlisting}
\centerline{Listing 2.3: Verkettete Aufrufe umfasst mit NULL-Prüfungen}

\vspace{0.3cm}

Autoboxing bezeichnet die mit Java 1.5 eingeführte automatische Umwandlung zwischen
primitiver Datentypen sowie deren Wrapper-Typen. Diese implizite Umwandlung wird
durch zusätzliche Methodenaufrufe durch den Compiler eingewebt und ist für den
Java-Interpreter nicht von normalen Aufrufen zu unterscheiden.

Für die manipulation des Bytecodes zur Verbesserung der Fehlertoleranz sollte
dies ebenfalls keinen Unterschied bieten.


\section{Bytecode-Analyse}

Mithilfe des im ASM enthaltenenen Textifier Programms können verschiedene Lösungswege
deassembliert und analysiert werden. Zum einfacheren Aufruf wurde ein kleines Shell-Script
(siehe textifier) erstellt. Mitdessen Hilfe wurden etwa für das in Listing 2.4 angegebene
Java-File die in 2.5 angegebene Ausgabe erzeugt.

Der Aufruf erfolgt über den Scriptnamen gefolgt von einer Java-Bytecode-Datei:

\texttt{./textifier target/test-classes/de/fhkoeln/gm/cui/javahardener/testcases/Test1.class}


\subsection{Ausgangsbasis}

\begin{lstlisting}[basicstyle=\ttfamily,numbers=left,numberstyle=\footnotesize\ttfamily,backgroundcolor=\color{source}]
package de.fhkoeln.gm.cui.javahardener.testcases;
public class Test1 {
	public int getStringLength(Map<String, String> map, String key) {
		return map.get(key).length();
	}
}
\end{lstlisting}
\centerline{Listing 2.4: Beispiel Sourcecode mit Null-Prüfung}

\vspace{0.3cm}


\begin{lstlisting}[basicstyle=\ttfamily,numbers=left,numberstyle=\footnotesize\ttfamily,backgroundcolor=\color{source}]
public class de/fhkoeln/gm/cui/javahardener/testcases/Test1 {
  public getStringLength(Ljava/util/Map;Ljava/lang/String;)I
    ALOAD 1
    ALOAD 2
    INVOKEINTERFACE java/util/Map.get (Ljava/lang/Object;)Ljava/lang/Object;
    CHECKCAST java/lang/String
    INVOKEVIRTUAL java/lang/String.length ()I
    IRETURN
    MAXSTACK = 2
    MAXLOCALS = 3
}
\end{lstlisting}
\centerline{Listing 2.5: Auszug ASM Assembler-Ausgabe für Listing 2.4}

\vspace{0.3cm}


Im folgenden sollen die Unterschiede aufgezeiugt werden, wenn man diese ursprüngliche
Version mit gegen NPE gesicherte Versionen vergleicht. Die dafür angelegten Klassen
befinden sich im test-Ordner innerhalb des Java-Packages
\texttt{de.fhkoeln.gm.cui.javahardener.analysebytecode}.

\subsection{Bedingungsoperator ?:}

Durch die Null-Prüfung mit einem Bedingungsoperator
(etwa \texttt{entry != null ? entry.toString() : null})
fügt der Compiler zwei Labels (Ziele für Springmarken) ein und prüft
anschließend die aktuell auf dem Stack liegende entry Variable (vgl. Listing 2.6 Zeile 1)
auf null (Z. 2). Ergebnis die \texttt{NULL}-Prüfung wahr springt die Ausführung
zur angegebenen Sprungmarke (hier L0) und fügt eine \texttt{NULL}-Referenz auf den Stack hinzu.
Falls die \texttt{NULL}-Prüfung falsch ergibt wird die Ausführung fortgesetzt
und der eigentliche Methodenaufruf durchgeführt (\texttt{INVOKEVIRTUAL} in Zeile 4).
Um anschließend den nicht benötigten Alternativen Zweig der Anwendung zu gehen wird
dieser mithilfe eines \texttt{GOTO}s (hier zur Sprungmarke L1) übersprungen.

\begin{lstlisting}[basicstyle=\ttfamily,numbers=left,numberstyle=\footnotesize\ttfamily,backgroundcolor=\color{source}]
    ALOAD 2 /* entry */
    IFNULL L0
    ALOAD 2 /* entry */
    INVOKEVIRTUAL java/lang/String.toString ()Ljava/lang/String;
    GOTO L1
   L0
    ACONST_NULL
   L1
\end{lstlisting}
\centerline{Listing 2.6: Auszug ASM für Null-Prüfung mit Bedingungsoperator}

\vspace{0.3cm}


\subsection{Try-Catch}

Eine weitere Möglichkeit wäre die mögliche Ausnahmebehandlung von dem eingebauten
try-catch Mechanismus behandeln zu lassen und einen entsprechenden Block um
den möglicherweise zu fehlern führenden Aufruf zu erstellen.

Für dieses Vorgehen wird eine zusätzliche lokale Variable benötigt,
welche im Fehlerfall mit einem Defaultwert gefüllt wird:

\texttt{int l; try { l = entry.length(); } catch (NPE e) { l = 0 }}

Der dadrauf entstehende Bytecode speichert das Ergebnis des Originalaufrufs
in einer lokalen Variable (Listing 2.7 Zeile 4 und 5). Sollte es während
dieses Aufrufs zu einer Fehlerbehandlung kommen wird diese Variable mit
einer \texttt{NULL}-Referenz überschrieben (Zeile 10 und 11).


\begin{lstlisting}[basicstyle=\ttfamily,numbers=left,numberstyle=\footnotesize\ttfamily,backgroundcolor=\color{source}]
    TRYCATCHBLOCK L0 L1 L2 java/lang/NullPointerException
   L0
    ALOAD 2
    INVOKEVIRTUAL java/lang/String.length ()I
    ISTORE 3
   L1
    GOTO L3
   L2
    ASTORE 4
    ICONST_0
    ISTORE 3
   L3
\end{lstlisting}
\centerline{Listing 2.7: Auszug ASM für Null-Prüfung mit try-catch}

\vspace{0.3cm}

Insgesamt fällt auf das dieser Code bereits bei diesem einfachem Beispiel deutlich
mehr Intruktionen beinhaltet als die zuvor genannte Bedingungsoperator-Variante.
Gleichzeitig wird für quasi jeden Methodenaufruf eine zusätzliche lokale Variable
benötigt. (Ggf. könnten diese auf eine Varaible je Datentyp kombiniert werden.)

Dadrüber hinaus würde diese Variante nicht nur unmittelbare NullPointerExceptions
abfangen sondern auch Fehler welche innerhalb der Methode ausgeführt werden und
ggf. gar nicht vom java-hardener manipuliert wurden.

