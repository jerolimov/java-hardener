%!TEX root = ../Dokumentation.tex

\chapter{Analyse}

%Um die Komplexität abschätzen zu können bedarf es der Vorherigen Analyse
%möglicher Anwendungsfälle oder Probleme.

\section{Problemstellung}

Wie in der Einführung beschrieben, können Objectaufrufe, z.B. durch
Methoden- und Variablenaufrufe (lesend und schreibend), auf \texttt{NULL}
durch vorheriges Prüfen gesichert werden.
Auch andere Fälle, etwa der Zugriff auf Arrays (\texttt{[index]}-Zugriff oder \texttt{.length}) kann
zu NPE-Ausnahmefehlern führen. Nicht alle diese Anwendungsfälle werden
in diesem Prototypem umgesetzt sollen aber wenigstens in dieser Einführung angesprochen werden.

Problematisch sind insbesondere verkettete Aufrufe (vgl. Listing 2.1).
So müssen die zwischen Ergebnisse etwa in lokalen Variablen gespeichert werden (vgl. Listing 2.2)
oder die Aufrufe wiederholt werden wenn diese in umgebende Bedingungen einzubauen (vgl. Listing 2.3).
Letzteres würde jedoch nicht nur die Performance negativ beeinflussen, sondern
könnte bei inmutablen Zugriffen auch zu Fehlerhaften Programmabläufen führen.

\begin{lstlisting}[basicstyle=\ttfamily,numbers=left,numberstyle=\footnotesize\ttfamily,backgroundcolor=\color{source}]
Deque<Map<String, Integer>> example = null;
int size = example.getFirst().get("size");
\end{lstlisting}
\centerline{Listing 2.1: Beispiel für verkette Aufrufe}

\vspace{0.3cm}

\begin{lstlisting}[basicstyle=\ttfamily,numbers=left,numberstyle=\footnotesize\ttfamily,backgroundcolor=\color{source}]
Deque<Map<String, Integer>> example = null;
Map v1 = example.getFirst();
Integer v2 = v1.getSize("size");
int size = v2 != null ? v2.intValue() : 0;
\end{lstlisting}
\centerline{Listing 2.2: Umwandlung verketteter Aufrufe in lokale Variablen}

\vspace{0.3cm}

\begin{lstlisting}[basicstyle=\ttfamily,numbers=left,numberstyle=\footnotesize\ttfamily,backgroundcolor=\color{source}]
Deque<Map<String, Integer>> example = null;
int size = 0;
if (example != null &&
	example.getFirst() != null &&
	example.getFirst().get("size") != null) {
	size = example.getFirst().get("size");
}
\end{lstlisting}
\centerline{Listing 2.3: Verkettete Aufrufe umfasst mit NULL-Prüfungen}

\vspace{0.3cm}

Autoboxing bezeichnet die mit Java 1.5 eingeführte automatische Umwandlung zwischen
primitiver Datentypen sowie deren Wrapper-Typen. Diese implizite Umwandlung wird
durch zusätzliche Methodenaufrufe durch den Compiler eingewebt und ist für den
Java-Interpreter nicht von normalen Aufrufen zu unterscheiden.

Für die manipulation des Bytecodes zur Verbesserung der Fehlertoleranz sollte
dies ebenfalls keinen Unterschied bieten.


\section{Bytecode-Analyse}

Die folgenden SHELL SCRIPT....



\begin{lstlisting}[basicstyle=\ttfamily,numbers=left,numberstyle=\footnotesize\ttfamily,backgroundcolor=\color{source}]
package de.fhkoeln.gm.cui.javahardener.testcases;
public class Test1 {
	public int getStringLength(Map<String, String> map, String key) {
		return map.get(key).length();
	}
}
\end{lstlisting}
\centerline{Listing 2.4: Beispiel Sourcecode mit Null-Prüfung}


\begin{lstlisting}[basicstyle=\ttfamily,numbers=left,numberstyle=\footnotesize\ttfamily,backgroundcolor=\color{source}]
public class de/fhkoeln/gm/cui/javahardener/testcases/Test1 {
  public getStringLength(Ljava/util/Map;Ljava/lang/String;)I
    ALOAD 1
    ALOAD 2
    INVOKEINTERFACE java/util/Map.get (Ljava/lang/Object;)Ljava/lang/Object;
    CHECKCAST java/lang/String
    INVOKEVIRTUAL java/lang/String.length ()I
    IRETURN
    MAXSTACK = 2
    MAXLOCALS = 3
}
\end{lstlisting}
\centerline{Listing 2.5: Auszug ASM Assembler-Ausgabe für Listing 2.4}


