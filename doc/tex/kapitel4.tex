%!TEX root = ../Dokumentation.tex

\chapter{Fazit}

\section{Projektergebnis}

Im Rahmen dieses Prototypes wurde gezeigt, dass auf Basis des Visitor Patterns
eine bequeme Anpassung von Java-Bytecode möglich ist. Allerdings bringt
dieses Vorgehen auch entsprechende Probleme mit sich und der fehlende
Ausführungskontext verhindert u.a., dass dieses Vorgehen auf Methodenaufrufe
mit mehr als einem Argument angewendet werden kann.


\section{Erweiterungsmöglichkeiten}

Auf Basis der gewonnenen Erkenntnis könnte dieses Vorgehen auch in kurzer Zeit
für weitere Situationen, in dennen \texttt{NullPointerExceptions} auftreten können,
erweitert werden:
Beispielsweise das Lesen (Instruktion \texttt{GETFIELD}) oder
Schreiben (\texttt{SETFIELD}) von Attributen.

Auch der Zugriff auf Arrays (\texttt{[B,C,S,I,F,L,D,A]ALOAD} bzw. \texttt{[B,C,S,I,F,L,D,A]ASTORE}))
bzw. das Auslesen der Array-Länge (\texttt{ARRAYLENGTH}) kann so gegen \texttt{NULL}-Zugriff gesichert werden.

Da die zusätzlichen Instruktionen die Performance einer Anwendung negativ beeinflussen
können, müssten noch weitere mögliche Optimierungspotenzialle erkannt werden. Dieser Kontext,
etwa Schleifen oder eine vorherige Zuweisung, ist auf Basis des Visitor Patterns
nicht oder nur sehr schwer zu fassen.
Für eine Bewertung der Performance wären Änderungsstatiken sowie Performanceanalysen nötig.

Darüber hinaus wären Optionen zum Aktivieren/Deaktivieren der Manipulationen auf Package,
Klassen oder Methoden Basis sinnvoll, etwa über Umgebungsvariablen.
So könnten Zugriffe auf häufige, aber im generellen als sicher geltene Aufrufe
unverändert bleiben (Z.B. \texttt{System.[in,out,err].*} - Aufrufe).

Ob eine \texttt{NULL}-Prüfung bereits durch den Programmier sichergestellt wurde
(oder durch ein doppeltes laden durch die hier entwickelten Klassen), läßt sich
aktuell nicht entscheiden. Auch, dass in gewissen Situationen eine weitere Prüfung
nicht nötig ist, kann auf der Basis der aktuellen Entwicklung nicht nachgerüstet werden.
So müssen Variablen Zuweisungen aus einem Konstruktur, Konstante Zuweisungen (etwa eines
primitiven Datentypes oder eines Strings) oder die für das Autoboxing
verwendeten Methoden \emph{WrapperKlasse}\texttt{.valueOf()} nicht geprüft werden.
Diese Zuweisungen Garantieren, dass eine Variable nicht-\texttt{NULL} sein kann.

%\section{Reflektion}


