%!TEX root = ../Dokumentation.tex

\chapter{Fazit}

\section{Projektergebnis}

Im Rahmen dieses Prototypes wurde gezeigt das auf Basis des Visitor-Patterns
eine bequeme Anpassung von Java-Bytecode möglich ist. Allerdings bringt
dieses Vorgehen auch entsprechende Probleme mit sich und der fehlende
Ausführungskontext verhindert u.a. das dieses Vorgehen auf Methodenaufrufe
mit mehr als einem Argument angewendet werden kann.


\section{Erweiterungsmöglichkeiten}

Auf Basis der gewonnen Erkenntnis könnte dieses Vorgehen auch in kurzes Zeit
für weitere NullPointerExceptions-Situationen erweitert werden: Etwa das
Lesen (Instruktion GETFIELD) oder Schreiben (SETFIELD) von Attributen.

Auch der Zugriff auf Arrays ([B,C,S,I,F,L,D,A]ALOAD bzw. [B,C,S,I,F,L,D,A]ASTORE))
bzw. die Längenprüfung (ARRAYLENGTH) gegen NULL-Zugriff gesichert werden.

Während die zusätzlichen Instruktionen die Performance einer Anwendung negativ beeinflussen
können wären mögliche Optimierungspotenzialle oft nur über einen großeren Kontext
(Vorherige Zuweisungen, Schleifen, etc.) zu fassen.
Für eine Bewertung der Performance wären Änderungsstatiken sowie Performanceanalysen
nötig.

Dadrüber hin wären Optionen zum aktivieren/deaktivieren des manipulation auf Package,
Klassen oder Methoden Basis. Etwa über Umgebungsvariablen.
So können etwa Zugriffe auf häufige, aber im generellen als \"sichere\" geltene Aufrufe
ungeprüft bleiben (etwa \texttt{System.[in,out,err].*}-Aufrufe).

Ob eine NULL-Prüfung etwa bereits durch den Programmier sichergestellt wurde
(oder durch ein doppeltes laden durch die hier entwickelten Klassen) läßt sich
aktuell nicht entscheiden. Auch das in gewissen Situationen eine weitere Prüfung
nicht nötig wäre könnte auf der Basis der aktuellen Entwicklung nicht nachgerüsstet werden.
(Variablen Zuweisungen aus einem Konstruktur, Konstante Zuweisungen (etwa eines
primitiven Datentypes oder eines Strings) oder die für das
Autoboxing verwendeten Methoden valueOf sind etwa Garantien dafür
dass eine Variable nicht NULL sein kann.)

%\section{Reflektion}


